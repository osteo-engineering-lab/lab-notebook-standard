\documentclass[11pt, oneside]{article}   	% use "amsart" instead of "article" for AMSLaTeX format

% --------------------------------------- Packages -----------------------------------------

\usepackage{geometry}                		% See geometry.pdf to learn the layout options. There are lots.
	\geometry{letterpaper}                   		% ... or a4paper or a5paper or ... 
	%\geometry{landscape}                		% Activate for for rotated page geometry
%\usepackage[parfill]{parskip}    		% Activate to begin paragraphs with an empty line rather than an indent
\usepackage{graphicx}				% Use pdf, png, jpg, or eps§ with pdflatex; use eps in DVI mode
								% TeX will automatically convert eps --> pdf in pdflatex		
\usepackage{amssymb}

\usepackage{lipsum}					% For including random text.
\usepackage{circuitikz}				% For building circuit diagrams.


% --------------------------------------- Path Setting -----------------------------------------

\graphicspath{{./Figures/}}				

% --------------------------------------- Title Matter -----------------------------------------

\title{Dean Lab Notebook Formatting}
\author{Lexi Cantu}
\date{2014-08-20}					% Activate to display a given date or no date

% --------------------------------------- Begin Document -----------------------------------------

\begin{document}
\maketitle 							% Places title matter

\tableofcontents{}					% Places table of contents

% --------------------------------------- Introduction -----------------------------------------
\newpage
\section*{Introduction}	
All lab notebooks are legal documents and property of Dean Lab. Notebooks should remain in the Biomedical Research Tower when not in use. In order to ensure they can be accurately referenced in the present and future, consistent formatting and labeling conventions are necessary. This document is intended to provide a complete guide to properly setting up and maintaining a lab notebook for Dean Lab.

\addcontentsline{toc}{section}{Introduction}	% Unnumbered sections are not included in table of contents automatically without this line

% --------------------------------------- Numbering -----------------------------------------

\section{Numbering}

Every page in the lab notebook must be numbered. The page on which the year appears will be numbered "1." Every page after that will be numbered sequentially (2, 3, 4, 5...) through the end of the notebook. Only the top side of the page (the side you write entries on) should be numbered for these pages. All pages before the first page with an entry should be numbered on both sides with lowercase roman numerals (i, ii, iii, iv...) starting on the top of the first page with 'i'. Numbers should be written in the upper right corners of pages. The roman numerals on the backs of the first few pages should go on the upper left corners. Do not circle, box, underline, or include decorative marks around the page numbers. 

%--------------------------------------- Front Cover -----------------------------------------

\section{Front Cover}					

\normalsize									
\subsection[Lined Portion]{Lined Portion}

The following information should be clearly printed in permanent marker or pen on the lined portion of the white box on the front cover of the lab notebook.
\begin{itemize}
	\item Line 1: Your full first and last name
	\item Line 2: Your personal phone number
	\item Line 3: Address of the lab (460 W. 12th Ave. Rm. 1009a Columbus OH, 43210)
	\item Line 4: Your serial number (convention explained below)
\end{itemize}

\subsection[Sticker]{Sticker}	

A Dean Lab sticker should be placed upright in the following four places:
\begin{itemize}
	\item Outer side of front cover: bottom right corner
	\item Inner side of front cover: bottom right corner
	\item Outer side of back cover: top right corner
	\item Inner side of back cover: top right corner
\end{itemize}
	
Note that your notebook's serial number should also appear in each of these places. More stickers can be found in the black storage box above and to the right of Eric's desk. 

\subsection[Serial Number Convention]{Serial Number Convention}		

Your serial number is all three of your initials followed by a dash, and the number (starting with 01, then increasing sequentially for each new notebook you use). If you have no middle initial, use 'X'. If you have multiple middle names, use the initial of the first. In the case that you have identical 3-letter initials as someone else in the lab, add a number (starting with 1 if you're the first to repeat those initials) before the dash. A list of serial numbers can be found on Box through the path Box Documents/Operations/Lab Notebooks/lab-notebook-registry.xlsx

							

% --------------------------------------- Table of Contents -----------------------------------------

\section{Table of Contents}

The first page of the lab notebook should be titled "Table of Contents." Entries regarding the same project or subject can be referenced in whole in this table. For example: if a mock sol fraction test was completed, the pages containing the protocol planning, data collection, observations and data analysis can all be recorded in this table as "Mock Sol Fraction Test." Any particularly important figures, data sets, or notes should also be distinguished in this table. Writing on the backs of pages is acceptable in this section.

% --------------------------------------- List of Abbreviations -----------------------------------------

\section{List of Abbreviations}

The second through fourth pages of the lab notebook should be reserved for defining any abbreviations used in the lab notebook. Abbreviating long or frequently used words can save time when writing in the lab notebook, but in any case that they are used, they must be clearly defined in this section. Even abbreviations that may seem obvious, such as 't' for 'time' should be explained here to avoid any possibility of confusion. When using abbreviations, ensure that no abbreviation is ever defined more than once. For example, 't' cannot be used to mean 'time' in some cases and 'temperature' in others. Writing on the backs of pages is acceptable in this section.

% --------------------------------------- Year -----------------------------------------

\section{Year}

The fifth page of the lab notebook should contain the year the notebook was started in. This is the only information that should appear on this page. Recall that this page is numbered '1'.

% --------------------------------------- Notebook Entries -----------------------------------------

\section{Notebook Entries}

Each notebook entry should start with the date in the format YYYY/MM/DD. Below the date, any necessary information can be recorded neatly and clearly on the lines in the notebook. If a large amount of space in the notebook is left blank, there should be a note near the space indicating that it was intentional. In this section of the notebook, the backs of the pages should not be written on. This space is reserved for any comments added at a later time. If an entry continues over onto a new page in the notebook, "Continued" should be written at the top of the new page. At the end of each entry, a horizontal line should be drawn and signed on. 

% --------------------------------------- List of References -----------------------------------------

\section{List of References}

The last page of the lab notebook should be reserved for recording the contact information of helpful people. These people may include collaborators, suppliers, ULAR staff, EHS staff, etc.


% --------------------------------------- END -----------------------------------------

\end{document}  
